%!TEX root = ../main.tex

\addchap{\acronymsPhrase}

\begin{acronym}[YTMMM]
\setlength{\itemsep}{-\parsep}

\acro{AXI}{Advanced Extensible Interface}
\acro{ALU}{Arithmetic Logical Unit}
\acro{CSR}{Control and Status Registers}
\acro{CU}{Control Unit}
\acro{DRA}{Direct Register Access}
\acro{FPGA}{Field Programmable Gate Array}
\acro{FSM}{Finite State Machine}
\acro{HDL}{Hardware Description Language}
\acro{IP}{Intellectual Property}
\acro{ISA}{Instruction Set Architecture}
\acro{M-Mode}{Machine Mode}
\acro{PC}{Programm Counter}
\acro{PMP}{Physical Memory Protection}
\acro{PMA}{Physical Memory Attributes}
\acro{RF}{Register File}
\acro{RISC}{Reduced Instruction Set Computer}
\acro{RV32I}{RISC-V 32 bit Integer}
\acro{S-Mode}{Supervisor Mode}
\acro{SISD}{Single Instruction Single Data}
\acro{U-Mode}{User Mode}
\acro{VHDL}{Very High Speed Integrated Circuit Hardware Description Language}
\acro{WARL}{Write Any Read Legal}



\end{acronym}
