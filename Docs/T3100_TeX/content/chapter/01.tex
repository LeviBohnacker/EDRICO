%!TEX root = ../../main.tex

\chapter{Introduction}
These days one of the key benchmarks for technology is processing speed and
calculation power. To realize mathematical operations and execute programs,
different platforms can be utilized. The most commonly used unit is the standard
processor consisting of transistors realized on silicium and other materials.
Another crucial technology that is gaining more attention is the so-called
\acf{FPGA}. The FPGA consists of logical units that can
be wired and configured individually for the required use-case. The advantage of
FPGA is that the speed of applications can be drastically increased since the
hardware will be very optimized for the specific application.
This project aims to develop a \acf{IP}-core based on the Open
Source Instruction Set \acs{RISC}-V. The goal is to build a reusable unit of logic that can
interpret compiled C-Code. The IP core is realized in the \ac{VHDL} language and will be deployed on a FPGA. IP Cores are used in every computer, phone and electronic device that requires to execute some computational function. The developers of these IP Cores are big
companies like Intel, ARM or AMD. These IP Cores and Instruction Sets are strictly
licensed and not available for everyone. For the development of an own IP Core the
Instruction Set is the main source of information and therefore the RISC-V
open-source Instruction Set is used for this project.

