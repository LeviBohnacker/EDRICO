%!TEX root = ../../main.tex
\chapter{Conclusion}
During the course of this project, a preliminary version of the proposed processor design was developed, implemented and partly verified. Using the V-Modell project management approach, the design was developed leveraging a top-down line of action. Step by step decreasing granularity of the design from Requirements, System Design, Architecture Design to finally Module Design and Implementation. \\
Accompanying documents are provided to each of the corresponding V-Modell design phases. These mark milestones in the project and provide a good overview of the different levels of abstraction.\\
\ac{EDRICO} implements the full \ac{RV32I} \ac{ISA} including the Zicsr instruction, running in machine-mode only. It is composed of a \textit{Control Unit}, \ac{ALU}, \textit{Register File}, \textit{Exception Control Unit}, \textit{PMP \& PMA checker} as well as a \ac{AXI4-Lite} master and slave interface. It executes one instruction at a time by running the fetch, decode and execute phases. All required exceptions as well as the timer and software interrupt are implemented and can be handled by the machine. \\
\ac{EDRICO} is shown to work in simulation with an average \ac{CPI} of 12,07. Especially the time required for memory access imposes a significant overhead to the execution.\\
Future improvements are best deployed on increasing memory performance, for example by adding a cache structure. The source code, documentation and accompanying documents can be found under the official github repository of \ac{EDRICO}: \underline{\href{https://github.com/LeviBohnacker/EDRICO}{Github Repository}}.
