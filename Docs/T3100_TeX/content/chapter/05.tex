%!TEX root = ../../main.tex
\chapter{Lessons Learned}
During the different phase of the project multiple issues arose from which important lessons for future projects or continuations of this project can be learned. The first problem that was encountered during this thesis surfaced due to underestimating the complexity of this project. Even though \ac{RISC} is short for Reduced Instruction Set Computer, this does not mean that the complexity of such a machine is trivial, especially for ongoing bachelor students with limited experience in digital design, processing architecture and \ac{VHDL}.\\
The next lesson is strongly related to the first one, but is not solely applicable to the development of a \ac{RISC} \ac{CPU}. Despite thoroughly planning every module and nearly every clock cycle of the design, multiple issues were encountered during test and verification. The lesson learned here is to always include extra time for verification, whilst planning the project time line. Two or three weeks are probably not enough to test a design that was developed over the course of approximately five months. \\
As a last item to lean from, the \ac{AXI4-Lite} master serves as an example. The design of it is quite complex and took a lot of time even though it is not very RISC-V related. Due to the development from scratch it includes only limited functionality, e.g. burst transactions are not supported. Sometimes it is better to use IP blocks for such generic components. This will decrease time required for development and allow the project to focus on other more important tasks, such as verification.